\documentclass[twocolumn]{article}

\usepackage{graphicx}
\usepackage{amsmath}
\usepackage{amsthm}
\usepackage{amssymb}
\usepackage{url}
\usepackage{multirow}
\usepackage{times}
\usepackage{fullpage}

\newcommand{\comment}[1]{}


\title{CS685: Group 05 \\
DoMInoS - Discovery of Meta-Information of Songs}
\author{
\begin{tabular}{ccc}
	Satvik Chauhan & Shubham Tulsiani  & Shyam Upadhyay \\
	\url{satvikc@iitk.ac.in} & \url{shubhtul@iitk.ac.in} & \url{shyamupa@iitk.ac.in} \\
	Dept. of CSE & Dept. of CSE & Dept. of CSE \\
	\multicolumn{3}{c}{Indian Institute of Technology, Kanpur}
\end{tabular}
}
\date{Mid-sem report \\	% replace by ``initial'' or ``final'' as appropriate
\today}	% replace by actual date of submission or \today
\begin{document}

\maketitle

\begin{abstract}
	%
	Abstract of the project.
	%
\end{abstract}

\section{Introduction}

Automatically classifying songs into categories based on year, mood, artist or
genre is widely studied topic in the field of Music Information Retrieval
(MIR). A lot of research is already done \cite{mgc2011} \cite{ada2006} for
this task, as creating accurate music classifiers are useful in fields
like music search, recommendation systems etc. The categorization usually
involves extracting important features from songs and using the extracted
feature vector in classification by standard machine learning algorithms. But
finding good feature set for a particular classification problem is a non-trivial
task \cite {ada2006} \cite{feature2005}.


\subsection{Problem Statement}

State the problem as clearly and as formally as possible.
Explain the notations, etc.
Explain the objectives, and all the inputs.

\subsection{Related Work}

Fill in all relevant past work.

\comment{

Can also comment out paragraphs, etc.

}

\section{Algorithm or Approach}

Details of the method.

Put in a pseudo-code, etc.
Explain with figures.

\comment{

Use the following format for figures:

\begin{figure}[t]
	\centering
	\includegraphics[width=0.95\columnwidth]{figure_file}
	\caption{This figure explains this.}
	\label{fig:block}
\end{figure}

And refer as Figure \ref{fig:block}.

}

\section{Results}

Details of results, in tabular and/or graphical formats.

\comment{

\begin{table}[t]
	\centering
	\begin{tabular}{|c||cc|}
		\hline
		Header 1 & Desc 1 & Desc 2 \\
		\hline
		\hline
		Row 1 & Data 1-1 & Data 1-2 \\
		Row 2 & Data 2-1 & Data 2-2 \\
		\hline
	\end{tabular}
	\caption{Table of results.}
	\label{tab:results}
\end{table}

And refer as Table \ref{tab:results}.

}

\section{Conclusions}

Clearly state the conclusions.

Also, outline the future work.

\begin{thebibliography}{99}
\bibitem{mgc2011} Yoko Anan, Kohei Hatano, Hideo Bannai and Masayuki
  Takeda. Music Genre Classification Using Similarity Functions. In
  \emph{Proceedings of the 12th International Conference on Music Information Retrieval}
(ISMIR'11).
\bibitem{ada2006} James Bergstra, Norman Casagrande, Dumitru Erhan,
Douglas Eck, and Balazs Kegl. Aggregate features and AdaBoost for music
classification. \emph{Machine Learning},
65:473–484, 2006.
\bibitem{feature2005} Thomas Lidy and Andreas Rauber. Evaluation of feature
  extractors and psycho-acoustic transformations for music genre
  classification. In \emph{Proceedings of the 6th International Conference on Music Information Retrieval}
(ISMIR'05), pages 34–41, 2005.
\end{thebibliography}

\end{document}
