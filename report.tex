\documentclass[twocolumn]{article}

\usepackage{graphicx}
\usepackage{amsmath}
\usepackage{amsthm}
\usepackage{amssymb}
\usepackage{url}
\usepackage{multirow}
\usepackage{times}
\usepackage{fullpage}

\newcommand{\comment}[1]{}

<<<<<<< HEAD
\title{CS685: Group 5 \\
DoMInoS - Discovery of Meta-Information of Songs}
\author{
\begin{tabular}{ccc}
	Satvik Chauhan & Shubham Tulsiani & Shyam Upadhyay \\
	\url{satvikc@iitk.ac.in} & \url{shubhtuls@iitk.ac.in} & \url{shyamupa@iitk.ac.in} \\
=======
\title{CS685: Group 05 \\
DoMInoS - Discovery of Meta-Information of Songs}
\author{
\begin{tabular}{ccc}
	Satvik Chauhan & Shubham Tulsiani  & Shyam Upadhyay \\
	\url{satvikc@iitk.ac.in} & \url{shubhtul@iitk.ac.in} & \url{shyamupa@iitk.ac.in} \\
>>>>>>> Added the file with names
	Dept. of CSE & Dept. of CSE & Dept. of CSE \\
	\multicolumn{3}{c}{Indian Institute of Technology, Kanpur}
\end{tabular}
}
\date{Mid-sem report \\	% replace by ``initial'' or ``final'' as appropriate
<<<<<<< HEAD
\today}	% replace by actual date of submission or \today
=======
%2nd Oct, 2012}	% replace by actual date of submission or \today
\today}
>>>>>>> Added the file with names

\begin{document}

\maketitle

\begin{abstract}
	%
	Abstract of the project.
	%
\end{abstract}

\section{Introduction}

Why is this important and interesting?

Please explain the backgroundas well.

Two example paragraphs follow.

In online social networks such as Facebook (\url{www.facebook.com}), Youtube
(\url{www.youtube.com}), Twitter (\url{www.twitter.com}), etc., people share
different content such as messages, videos, songs, opinions, blogs, etc.  In
another important form of networks---peer-to-peer networks---files are shared.
In both these cases, reputation of a peer matters; otherwise a user may get
exposed to objectionable content such as a virus.  Thus, the trust that other
users impart on a node is an important attribute of it.  Slashdot
(\url{www.slashdot.org}) and Epinions (\url{www.epinions.com}) are networks
where explicit opinions of a user as trust and distrust are available.

A network based on trust is quite different from other networks. An explicit
link in a network such as Facebook, Youtube signifies that two nodes are close.
However, in a trust based network, two nodes may be close and may be connected
but the link may show distrust.  More importantly, a neutral opinion in a trust
based network is quite different from a no-connection.  Consider a simple
example where node $A$ has $1000$ neutral opinions and $10$ negative opinions
about it and another node $B$ has only $10$ negative opinions about it.  It is
intuitive that node $A$ has a higher reputation.  However, if the neutral
opinion is modeled by the absence of an edge, the two nodes behave the same.
This indicates that neutral opinion is not the same as a no-connection.  In
other words, an edge with $0$ weight is different from an edge that is absent.

\subsection{Problem Statement}

State the problem as clearly and as formally as possible.
Explain the notations, etc.
Explain the objectives, and all the inputs.

\subsection{Related Work}

Fill in all relevant past work.

\comment{

Can also comment out paragraphs, etc.

}

\section{Algorithm or Approach}

Details of the method.

Put in a pseudo-code, etc.
Explain with figures.

\comment{

Use the following format for figures:

\begin{figure}[t]
	\centering
	\includegraphics[width=0.95\columnwidth]{figure_file}
	\caption{This figure explains this.}
	\label{fig:block}
\end{figure}

And refer as Figure \ref{fig:block}.

}

\section{Results}

Details of results, in tabular and/or graphical formats.

\comment{

\begin{table}[t]
	\centering
	\begin{tabular}{|c||cc|}
		\hline
		Header 1 & Desc 1 & Desc 2 \\
		\hline
		\hline
		Row 1 & Data 1-1 & Data 1-2 \\
		Row 2 & Data 2-1 & Data 2-2 \\
		\hline
	\end{tabular}
	\caption{Table of results.}
	\label{tab:results}
\end{table}

And refer as Table \ref{tab:results}.

}

\section{Conclusions}

Clearly state the conclusions.

Also, outline the future work.

\section*{References}

Directly type in bib entries.

Better is to use bibtex.

\end{document}
